\documentclass[a4paper,11pt]{report}
\usepackage[code]{mystandard}


\title{LINMA1702 modèles et méthodes d'optimisation : 
Logistique de la vaccination en Belgique
}
\author{Groupe 6 : Théau Lepouttre, Eliott Van Dieren et Nicolas Mil-Homens Cavaco}
\date{2020-2021}

\begin{document}
\maketitle
\section{Premier jet}

\subsection{Question 1.1}

Soit les variables suivantes,
\begin{itemize}
\item[$\bullet$]  $x \equiv$ le nombre de doses qui arrivent au centre de vaccination; 
\item[$\bullet$]  $y_i \equiv$ le nombre de doses administrées à la classe d'âge $i$;
\end{itemize}


contraintes par les paramètres suivants,
\begin{itemize}
\item[$\bullet$]  $b_l \equiv$ le nombre maximal de vaccins livrés par jour;
\item[$\bullet$]  $b_v \equiv$ le nombre maximal de vaccins administrés par jour;
\item[$\bullet$]  $c_t \equiv$ le coût de transport;
\item[$\bullet$]  $c_v \equiv$ le coût de vaccination;
\item[$\bullet$]  $c_{tot} \equiv$ le budget total disponible;
\item[$\bullet$]  $n_s \equiv$ le nombre de personnes susceptibles;
\item[$\bullet$]  $n_{nv} \equiv$ le nombre de personnes qui ne veulent pas se faire vacciner parmi les suceptibles. 
\end{itemize}
Puisque dans un premier temps, on suppose que le stockage des vaccins est impossible, la totalité de vaccins livrés doivent être administrés ou doivent être jetés, c'est-à-dire que $x \geq y$.

Le problème s'écrit donc

\begin{eqnarray*}
\max_{x\ \in \mathbb{R}, y\ \in \mathbb{R}^m}& \sum_{i=1}^{m} y_i & \\
x - \sum_{i=1}^{m} y_i &\geq& 0 \\
c_t\ x + c_v \sum_{i=1}^{m} y_i &\leq& c_{tot} \\
x &\leq& b_l \\
\sum_{i=1}^{m} y_i &\leq&\ b_v \\
y_i &\leq& (n_s)_i - (n_{nv})_i \\
x,\ y &\geq& 0
\end{eqnarray*}

\newpage
\lstset{style=mystyle}

\begin{lstlisting}[language=Python]
import numpy as np
import matplotlib.pyplot as plt
from mip import *

# Donnees du probleme
m     = 6    # nombre de classes d'age
c_t   = 1    # Prix de livraison d'un vaccin
c_v   = 1    # prix d'administration d'un vaccin
c_tot = 100  # budget total autorise
b_l   = 100  # nombre maximal de vaccins livres par jour
b_v   = 150  # nombre maximal de vaccins administres par jour

# nombre de personnes suceptibles dans chaque classe d'age
n_s   = np.ones(m) * 20   
# nombre de personnes ne voulant pas etre vaccinees dans chaque classe d'age
n_nv  = np.ones(m) * 1     

model = Model('centre unique', sense=MAXIMIZE, solver_name=CBC)

# Variables
x = model.add_var(var_type=INTEGER, lb=0)                               
y = np.array([model.add_var(var_type=INTEGER, lb=0) for i in range(m)])

# Objectif
model.objective = maximize(xsum(y))

# Contraintes
model += c_t * x + xsum(c_v * y) <= c_tot
model += x - xsum(y) >= 0
model += x <= b_l
model += xsum(y) <= b_v
for i in range(m):
    model += y[i] <= n_s[i] - n_nv[i]

model.optimize()

print(f"x = {x.x}")
print(f"y = {[y[i].x for i in range(m)]}")
print(f"f(x,y) = {model.objective_value}")

\end{lstlisting}

\end{document}