\documentclass[a4paper,11pt]{report}
\usepackage[code]{mystandard}


\title{LINMA1702 modèles et méthodes d'optimisation : 
Logistique de la vaccination en Belgique
}
\author{Groupe 6 : Théau Lepouttre, Eliott Van Dieren et Nicolas Mil-Homens Cavaco}
\date{2020-2021}

\begin{document}
\maketitle
\section{Premier jet}

\subsection{Hypothèses}
On considère un certaine population répartie en $m$ classes d'âge dans $n$ provinces en Belgique.
Parmi les personnes constituant cette population, on distingue $2$ parties disjointes:   

\begin{itemize}
\item[$\bullet$]  les personnes susceptibles;
\item[$\bullet$]  les autres, c'est-à-dire les malades, les vaccinés, les guéris et les morts.
\end{itemize}

On suppose cette population comme constante dans le temps.
Notre objectif est de minimiser le nombre de morts dû à l'épidémie.

\subsection{Question 1.1}

Dans ce premier modèle, on considère le cas d'un seul centre ($n=1$).
En outre, le stockage des vaccins n'est pas autorisé.
On suppose également que la durée de la campagne de vaccination s'étant sur une durée de $T$ jours.

Soit les variables suivantes
\begin{itemize}
\item[$\bullet$] $x^t \equiv$ le nombre de doses qui arrivent au centre de vaccination au temps $t$; 
\item[$\bullet$] $y_i^t \equiv$ le nombre de doses administrées à la classe d'âge $i$ au temps $t$.
\end{itemize}

contraintes par les paramètres suivants,

contraintes par les paramètres suivants,

\begin{itemize}
\item[$\bullet$] $b_c^t \equiv$ le nombre de vaccins livrés dans l'entrepôt central;
\item[$\bullet$] $b_l^t \equiv$ le nombre maximal de vaccins livrés au temps $t$;
\item[$\bullet$] $b_v^t \equiv$ le nombre maximal de vaccins administrés au temps $t$;
\item[$\bullet$] $c_{tr} \equiv$ le coût de transport;
\item[$\bullet$] $c_v \equiv$ le coût de vaccination;
\item[$\bullet$] $c_{tot} \equiv$ le budget total disponible.
\end{itemize}


Soit 
\begin{itemize}
\item[$\bullet$]  $\lambda_i^t \equiv$ la fraction parmi les personnes susceptibles tombant malade au jour $t$ pour la classe $i$;
\item[$\bullet$] $\varepsilon_i^t \equiv$ la fraction parmi les personnes de la classe $i$ tombées malades au jour $t$ qui décèderont;
\item[$\bullet$] $\mu_i \equiv$ la fraction de population disposée et autorisée à se faire vacciner.
\end{itemize}

\begin{itemize}
\item[$\bullet$]   $(n_s)_i^t \equiv$ le nombre de personnes susceptibles parmi les personnes de la classe $i$ au temps $t$;
\item[$\bullet$]  $(n_m)_i^t \equiv$ le nombre de personnes de la classe $i$ qui sont tombées malade au temps $t$;
\item[$\bullet$]  $(n_v)_i^t \equiv$ le nombre de personnes vaccinées pour une classe $i$ au temps $t$.
\end{itemize}

On a 
\begin{eqnarray*}
(n_v)_i^t   &=& y_i^t \\
(n_m)_i^t &=& \lambda_i^t\ (n_s)_i^t \\
(n_s)_i^t  &=& (n_s)_i^{t-1} - (n_m)_i^{t-1} - (n_v)_i^{t-1} \\
                  &=& (1-\lambda_i^{t-1})\ (n_s)_i^{t-1} - y_i^{t-1}
\end{eqnarray*}


On identifie les contraintes suivantes:
\begin{itemize}
\item[$\bullet$] Le budget total de la campagne de vaccination est limité.
\item[$\bullet$] Le total de vaccins administrés au temps $t$ ne peut pas dépasser le nombre total de vaccins livrés la veille, et ce sur toute la durée de la campagne.
\item[$\bullet$] Le nombre de vaccins livrés est limité chaque jour par le nombre de vaccins disponibles dans l'entrepôt central.
\item[$\bullet$] Le nombre de vaccins livrés est limité chaque jour par la limite du centre de vaccination.
\item[$\bullet$] Le nombre de vaccins administrés est limité chaque jour.
\item[$\bullet$]  Seules les personnes disposées et autorisées peuvent se faire vacciner (18+ et consentant). 
\end{itemize}

Finalement, on prend la convention que $x^0 = 0 = x^T$, en d'autres termes aucun vaccin n'a été livré la veille du début de la campagne ou ne doit être livré le dernier jour.

Le problème s'écrit donc

\begin{eqnarray*}
\min_{x, y} & \sum_{t=1}^{T} \sum_{i=1}^{m}  \varepsilon_i^t\ \lambda_i^t\ (n_s)_i^t \\
\sum_{t=1}^T \left(c_{tr}\ x^t + c_v \sum_{i=1}^{m} y_i^t\right)  & \leq & c_{tot} \\
x^{t-1} - \sum_{i=1}^{m} y_i^t & \geq & 0\\
x^{t} &\leq & b_c^t \qquad\\  
x^t    &\leq & b_l^t \qquad \\
\sum_{i=1}^{m} y_i^t & \leq &\ b_v^t \\
\sum_{k=1}^t y_i^k & \leq & \mu_i\ (n_s)_i^0\\
x,\ y & \geq & 0
\end{eqnarray*}

%\newpage
%\lstset{style=mystyle}

%\lstinputlisting[language=Python]{Version_locale/Q1.1.py}

\subsection{Question 1.2}


On considère à présent le cas de $n$ provinces. Dans le cas de la Belgique, on aura $n=10$.
On autorise également le stockage des vaccins dans les centres.

Par rapport au modèle précédent, on introduit 
\begin{itemize}
\item[$\bullet$] $z_j^t \equiv$ le nombre de vaccins stockés au temps $t$ dans le centre de la province $j$, variable. Ces vaccins ne sont donc pas administrés au jour $t$.
\item[$\bullet$] $(c_s)_j \equiv$ le coût associé au stockage d'un vaccin dans le centre $j$, connu.
\end{itemize}

Outre les contraintes identifiées plus haut, on ajoute le fait que tous les vaccins qui sont arrivés la veille ou qui étaient en stock la veille peuvent soit être administrés le lendemain, soit être mis en stock le lendemain.

Finalement, puisqu'on suppose que chaque province $j$ subit l'épidémie de manière indépendante, les contraintes sont propres à chacune, excepté les contraintes de coût total et de vaccins disponibles au hangar central.

De manière similaire qu'à la question 1.1, on prend la convention:
\begin{itemize}
\item[$\bullet$] $x^0 = 0 = x^T$, car aucun vaccin n'a été livré la veille du début de la campagne ou ne doit être livré le dernier jour.
\item[$\bullet$] $z^0 = 0 = z^T$, car aucun vaccin n'est disponible en stock la veille du début de la campagne ou ne doit être mis en stock le dernier jour.
\end{itemize}
Le problème s'écrit donc

\begin{eqnarray*}
\min_{x, y, z}& \sum_{t=1}^{T} \sum_{i=1}^{m} \sum_{j=1}^{n} \varepsilon_{ij}^t\ \lambda_{ij}^t (n_s)_{ij}^t & \\
\sum_{t=1}^T \sum_{j=1}^{n} \left(c_{tr}\ x_j^t + c_v \sum_{i=1}^{m} y_{ij}^t + c_s z_j^t\right) &\leq& c_{tot} \\
x_j^{t-1} + z_j^{t-1} - \sum_{i=1}^{m} y_{ij}^t - z_j^{t} &\geq& 0 \\
\sum_{j=1}^n x_j^{t} &\leq & b_c^t \\ 
x_j^t &\leq& (b_l)_j^t \\
\sum_{i=1}^{m} y_{ij}^t &\leq&\ (b_v)_j^t \\
\sum_{k=1}^t y_{ij}^k &\leq& \mu_{ij}\ (n_s)_{ij}^0 \\
x,\ y,\ z &\geq& 0
\end{eqnarray*}

%\newpage
%\lstset{style=mystyle}

%\lstinputlisting[language=Python]{Version_locale/Q1.2.py}

\end{document}